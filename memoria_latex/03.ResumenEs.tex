\cleardoublepage
\chapter*{Resumen} % si no queremos que añada la palabra "Capitulo"
\addcontentsline{toc}{chapter}{Resumen} % si queremos que aparezca en el índice
\markboth{}{} % encabezado

	Este trabajo fin de grado (TFG) presenta el diseño e implementación de un sistema de actuación con el que se pueda controlar la refrigeración de una sala de servidores o de un centro de datos (CPD).

	En este trabajo se pretende dar soporte a un sistema de monitorización y optimización energética de centros de datos y poder aplicar una de sus optimizaciones que está basada en el sistema de refigeración. El sistema de optimización, mediante un algoritmo, predice la temperatura a la que debe estar una sala de servidores para que el consumo energético de dicha sala sea mínimo. Para lograr este objetivo, es necesario actuar sobre el sistema de refrigeración y regular su temperatura de funcionamiento de forma dinámica. De este modo, se consigue que la sala alcance el valor de temperatura deseado.

	Teniendo en cuenta esta necesidad, se ha diseñado e implementado un sistema de actuación basado en un sistema de control en lazo cerrado. Se ha conseguido que el sistema sea capaz de regular la temperatura de la sala, siguiendo el valor de temperatura óptima proporcionada por el algoritmo. Este sistema obtiene los datos de temperatura desde una plataforma web, en tiempo real y sin necesidad de ningún tipo de acción humana. También se ha conseguido que el sistema sea dinámico y capaz de responder a las distintas variaciones que puedan producirse en la temperatura óptima, de una forma estable y con la mayor rapidez posible. El sistema de control utilizado se basa en un controlador PID, que posee una implementación sencilla y proporciona un amplio rango de operación, además de que es ampliamente usado en la industria para el control de procesos, incluyendo la temperatura.

	Este sistema de actuación ha sido probado en la sala de servidores B039 del departamento de Ingeniería Electrónica y posteriormente será integrado en el sistema de monitorización y optimización energética desarrollado por el grupo \textit{GreenLSI}, perteneciente a este departamento.

\noindent\textbf{Palabras clave} 

\noindent Centro de procesamiento de datos, sistema de control, optimización energética, control de la temperatura, controlador PID, actuador, sistema ciber-físico, sistema de monitorización.

