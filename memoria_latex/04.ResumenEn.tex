\cleardoublepage
\chapter*{Abstract} % si no queremos que añada la palabra "Capitulo"
\addcontentsline{toc}{chapter}{Abstract} % si queremos que aparezca en el índice
\markboth{}{} % encabezado

	This final project work shows the design and implementation of an actuator to control the cooling system of a server room or data center.

	The objective of this work is to apply one type of energy optimization based on the cooling system. An energy monitoring and optimization system predicts what is the optimal temperature of the server room so that the energy consumption is minimum. To apply this optimization, it is necessary to modify dynamically the setpoint of the cooling system to achieve that the temperature of the room reaches the optimal temperature.

	For this reason, an actuator based on a closed loop control system has been designed and implemented. This actuator is capable to setting the optimal temperature in the server room or data center. The actuator obtains the temperature data connecting to a web plataform, in a automatic way and in real time. Also, the actuator is capable of responding to any type of optimum temperature in a dynamic, stable and fast way. The control system used by the actuator is based on a PID controller, which has a simple implementation and provides a wide range of operations. Besides, this type of controller is common used in the industry to process control, including the temperature.

	This actuator has been tested in a server room of the Electronic Engineering Department of Universidad Politécnica de Madrid. Later, this actuator will be integrated in the energy monitoring and optimization system developed by this group.

\noindent\textbf{Keywords}

\noindent Data centers, control system, actuator, energy optimization, temperature control, PID controller, Cyber-Physical System, monitoring system.



